\section*{Jetan Evaluation Function (50 Points)}

As part of the Jetan project, we discussed several
ideas for elements of evaluation functions.
You were required to implement a couple of
evaluation functions as part of the assignments.
Here's an idea for an additional heuristic.

Being in the middle of the board could be advantageous, as more moves
are available to the pieces in the middle.  We'll encourage pieces
to move to the middle by assigning value to the location of the
piece.  The value will be 12, minus the Manhattan distance
from the center of the board (4.5,4.5) to the piece's location.
For example,  a piece located at position (2,9) will have
a value of 12 - (abs(2-4.5)+abs(9-4.5)) = 12 - (2.5+4.5) = 5.

So, the evaluation function will calculate this value for your
pieces, and subtract from it this value for your opponent's pieces.
  
\subsection*{Task}

Add this evaluation function to your agent.

Play your agent against itself where both agents use this evaluation function.
Play your agent against itself where one agent uses this evaluation function,
and the other uses your normal evaluation function.  That's a total of 2 games.

Write a brief report with the results of the games, and your observation on the
motion of the pieces under this evaluation function. The report must
also include instructions on running your agent with this evaluation function.
Submit the report to the Canvas task.

Be sure that your code with this evaluation function is pushed to github.


