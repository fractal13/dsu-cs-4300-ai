% !TEX program = pdflatex
\documentclass[aspectratio=169]{beamer}

% Shared course configuration (theme, colors, metadata, macros)
\usepackage{agentsslides}
\usepackage{multicol} % for two-column TOCs


\begin{document}

% ---------- Title ----------
\begin{frame}
  \titlepage
\end{frame}



% ================================
% Day 1 Slides: Background & Problem Formulation
% ================================

\section{Uninformed Search: Day 1}

\begin{frame}{Motivation for Search}
  \begin{itemize}
    \item Agents often don’t know solutions in advance.
    \item They must \textbf{explore possible actions and consequences}.
    \item Search provides a systematic way to do this.
  \end{itemize}
  \vspace{0.3cm}
  \begin{center}
    \fig[width=0.3\textwidth]{maze} % placeholder maze image
  \end{center}
\end{frame}

\begin{frame}{Examples of Search in Everyday AI}
  \begin{itemize}
    \item Navigation (GPS directions)
    \item Solving puzzles (Sudoku, Rubik’s cube)
    \item Planning tasks (robotics, scheduling)
  \end{itemize}
\end{frame}

% --- Introduce examples
\begin{frame}{Example 1: Vacuum World (Toy Problem)}
  \begin{minipage}{0.55\textwidth}
    \begin{itemize}
      \item Agent in two rooms (A and B).
      \item Rooms can be clean or dirty.
      \item Goal: both rooms clean.
    \end{itemize}
  \end{minipage}%
  \begin{minipage}{0.43\textwidth}
    \fig[width=\textwidth]{vacuum} % simple 2-room diagram
  \end{minipage}
\end{frame}

\begin{frame}{Example 2: Route Finding (Medium Problem)}
  \begin{minipage}{0.55\textwidth}
    \begin{itemize}
      \item States = cities (nodes).
      \item Actions = roads connecting cities.
      \item Goal: travel from start city to destination.
    \end{itemize}
  \end{minipage}%
  \begin{minipage}{0.43\textwidth}
    \fig[width=\textwidth]{romania_map} % placeholder Romania map
  \end{minipage}
\end{frame}

\begin{frame}{Example 3: Cryptarithmetic (Complex Problem)}
  \begin{minipage}{0.55\textwidth}
    \begin{itemize}
      \item Puzzle: SEND + MORE = MONEY.
      \item States = partial digit assignments.
      \item Goal: complete valid assignment.
    \end{itemize}
  \end{minipage}%
  \begin{minipage}{0.43\textwidth}
    \fig[width=\textwidth]{cryptarith} % stylized equation graphic
  \end{minipage}
\end{frame}

% --- Core concepts
\begin{frame}{Defining a State}
  \begin{itemize}
    \item A \textbf{state} = description of the current situation.
    \item Should capture all relevant information to decide what to do next.
  \end{itemize}
  \vspace{0.3cm}
  \textbf{Think-Pair-Share:} How would you define a state in:
  \begin{itemize}
    \item Vacuum World?
    \item Route Finding?
    \item Cryptarithmetic?
  \end{itemize}
\end{frame}

\begin{frame}{Initial and Goal States}
  \begin{itemize}
    \item \textbf{Initial state}: where the agent starts.
    \item \textbf{Goal test}: condition to check for success.
  \end{itemize}
  \vspace{0.3cm}
  \textbf{Prompt:} What are the initial and goal states in each example?
\end{frame}

\begin{frame}{Actions and Successor Functions}
  \begin{itemize}
    \item \textbf{Actions}: available choices at a state.
    \item \textbf{Successor function}: mapping from state + action $\to$ new state.
  \end{itemize}
  \vspace{0.3cm}
  \textbf{Prompt:} List actions in Vacuum World, Route Finding, Cryptarithmetic.
\end{frame}

\begin{frame}{Transition Models}
  \begin{itemize}
    \item \textbf{Transition model}: describes what happens when an action is taken.
    \item Can be deterministic (predictable) or nondeterministic (uncertain).
  \end{itemize}
  \vspace{0.3cm}
  Example: driving between cities (sometimes nondeterministic: traffic, weather).
\end{frame}

\begin{frame}{Path Costs}
  \begin{itemize}
    \item \textbf{Path cost}: numerical value for a sequence of actions.
    \item Defines solution quality (shortest, cheapest, fastest).
  \end{itemize}
  \vspace{0.3cm}
  \textbf{Prompt:} What is a natural path cost in each of our three examples?
\end{frame}

\begin{frame}{Problem Formulation Recap}
  A search problem is defined by 5 components:

  \begin{enumerate}
    \item \textbf{Initial state:} $s_0$ \\
          (the starting point of the search)

    \item \textbf{Actions:} $A(s) \;\to\; \{a_1, a_2, \dots\}$ \\
          Returns the set of possible actions in state $s$

    \item \textbf{Transition model:} $T(s,a) \;\to\; s'$ \\
          Returns the resulting state when action $a$ is applied in state $s$

    \item \textbf{Goal test:} $G(s) \;\to\; \{\text{true}, \text{false}\}$ \\
          Checks whether state $s$ is a goal state

    \item \textbf{Path cost:} $C(s,a,s') \;\to\; \mathbb{R}_{\geq 0}$ \\
          Assigns a numeric cost to the step from $s$ to $s'$ via $a$
  \end{enumerate}

  \vspace{0.2cm}
  \textbf{Prompt:} How do these functions look in our 3 examples?
\end{frame}

\begin{frame}{Search Trees vs. Graphs}
  \begin{minipage}{0.55\textwidth}
    \begin{itemize}
      \item \textbf{Tree search}: may revisit states repeatedly.
      \item \textbf{Graph search}: avoids repeated states.
      \item Important for efficiency and correctness.
    \end{itemize}
  \end{minipage}%
  \begin{minipage}{0.43\textwidth}
    \fig[width=\textwidth]{tree_vs_graph}
  \end{minipage}
\end{frame}

\begin{frame}{Measuring Search Performance}
  Evaluation criteria:
  \begin{itemize}
    \item \textbf{Completeness}: guaranteed to find solution? 
    \item \textbf{Optimality}: guaranteed to find best solution? 
    \item \textbf{Time complexity}: how long? 
    \item \textbf{Space complexity}: how much memory? 
  \end{itemize}
\end{frame}

\begin{frame}{Day 1 Wrap-Up}
  \begin{itemize}
    \item Problem formulation = defining states, actions, goals, costs.
    \item Performance measured by completeness, optimality, time, space.
    \item Next time: algorithms that actually search.
  \end{itemize}
\end{frame}

% ================================
% Day 2 Slides: Uninformed Search Algorithms
% ================================

\section{Uninformed Search: Day 2}

\begin{frame}{From Formulation to Algorithms}
  \begin{itemize}
    \item Now that we know how to define a search problem…
    \item Let’s look at systematic strategies for exploring the state space.
  \end{itemize}
\end{frame}

% BFS
\begin{frame}{Breadth-First Search (BFS): Intuition}
  \begin{minipage}{0.55\textwidth}
    \begin{itemize}
      \item Expand shallowest nodes first.
      \item Explore all nodes at depth $d$ before $d+1$.
    \end{itemize}
  \end{minipage}%
  \begin{minipage}{0.43\textwidth}
    \fig[width=\textwidth]{bfs_levels}
  \end{minipage}
\end{frame}

\begin{frame}{BFS Example: Route Finding}
  \fig[width=0.8\textwidth]{bfs_route} % show BFS expansion on small map
\end{frame}

\begin{frame}{BFS Properties}
  \begin{itemize}
    \item Complete (if branching factor finite).
    \item Optimal for uniform step costs.
    \item Time/space complexity: $O(b^{d})$.
  \end{itemize}
\end{frame}

% UCS
\begin{frame}{Uniform-Cost Search (UCS): Intuition}
  \begin{itemize}
    \item Expand lowest-cost node first.
    \item Generalizes BFS for non-uniform costs.
  \end{itemize}
\end{frame}

\begin{frame}{UCS Example: Route Finding}
  \fig[width=0.8\textwidth]{ucs_route}
\end{frame}

\begin{frame}{UCS Properties}
  \begin{itemize}
    \item Complete if step costs $\geq \epsilon$.
    \item Optimal (finds cheapest path).
    \item Time/space similar to BFS.
  \end{itemize}
\end{frame}

% DFS
\begin{frame}{Depth-First Search (DFS): Intuition}
  \fig[width=0.7\textwidth]{dfs_tree}
  \begin{itemize}
    \item Go deep into one branch, then backtrack.
  \end{itemize}
\end{frame}

\begin{frame}{DFS Example: Vacuum World}
  \fig[width=0.8\textwidth]{dfs_vacuum}
\end{frame}

\begin{frame}{DFS Properties}
  \begin{itemize}
    \item Not complete (can get stuck in infinite branch).
    \item Not optimal.
    \item Space efficient: $O(bm)$ vs. $O(b^d)$.
  \end{itemize}
\end{frame}

% DLS + IDDFS
\begin{frame}{Depth-Limited Search}
  \begin{itemize}
    \item DFS with a cutoff at depth $l$.
    \item Avoids infinite paths.
    \item May miss solution if $l$ too small.
  \end{itemize}
\end{frame}

\begin{frame}{Iterative Deepening DFS (IDDFS)}
  \begin{itemize}
    \item Repeat DFS with increasing depth limit.
    \item Combines BFS optimality with DFS space efficiency.
  \end{itemize}
  \fig[width=0.7\textwidth]{iddfs}
\end{frame}

% Comparisons
\begin{frame}{Comparison of Algorithms}
  \fig[width=\textwidth]{search_comparison_table}
\end{frame}

\begin{frame}{Complex Example: Cryptarithmetic}
  \begin{itemize}
    \item State space is huge.
    \item Branching factor: many digit assignments per step.
    \item Uninformed search quickly becomes impractical.
  \end{itemize}
\end{frame}

\begin{frame}{Day 2 Wrap-Up}
  \begin{itemize}
    \item Uninformed algorithms: BFS, UCS, DFS, DLS, IDDFS.
    \item Tradeoffs in completeness, optimality, efficiency.
    \item Motivation: we need \textbf{heuristics} to go further.
  \end{itemize}
\end{frame}







\end{document}
